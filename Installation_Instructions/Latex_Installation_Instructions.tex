\documentclass[a4paper,twoside,12pt,fleqn,openright]{article}
\setcounter{tocdepth}{4}
\def\pdfshellescape{1}
\usepackage{amsmath}
\usepackage{amsfonts}
\usepackage{graphicx}
\usepackage{a4wide}
% For Print version
%\usepackage[pdftex, plainpages=false, pdfpagelabels, pdfpagelayout=useoutlines, bookmarks, bookmarksopen=true, bookmarksnumbered=true, breaklinks=true, linktocpage, pagebackref=false, colorlinks=true, linkcolor=black, urlcolor=black, citecolor=black, anchorcolor=black, hyperindex=false, hyperfigures]{hyperref}
% For PDF Online version
\usepackage[pdftex, plainpages=false, pdfpagelabels, pdfpagelayout=useoutlines, bookmarks, bookmarksopen=true, bookmarksnumbered=true, breaklinks=true, linktocpage, pagebackref=false, colorlinks=true, linkcolor=blue, urlcolor=blue, citecolor=blue, anchorcolor=green, hyperindex=false, hyperfigures]{hyperref}
\usepackage{color}
\usepackage{textcomp}
\usepackage{amssymb}%
\usepackage{setspace}
\usepackage{tabularx}
\usepackage{etex}
\usepackage{fancyhdr}   % Use fancy headers package to produce headers
\usepackage{multicol}
\usepackage{multirow}
\usepackage{lscape}
\usepackage{rotating}
\usepackage{wrapfig}
\usepackage{float}
\usepackage{longtable}
\usepackage{subfigure}
\usepackage[figurename=Fig.,labelsep=space,tableposition=top]{caption}
\usepackage[round,colon,authoryear]{natbib}
\usepackage{appendix}
\usepackage{caption}
\usepackage{epstopdf}
\usepackage{url}
\usepackage[table]{xcolor}

%% Define a new 'leo' style for the package that will use a smaller font.
\makeatletter
\def\url@leostyle{%
  \@ifundefined{selectfont}{\def\UrlFont{\sf}}{\def\UrlFont{\small\ttfamily}}}
\makeatother
%% Now actually use the newly defined style.
\urlstyle{leo}
%*********************************** To copy ligatures ********************************************************************************************* %
\usepackage[ansinew]{inputenc}
\usepackage[T1]{fontenc}
%\usepackage{libertine}
\input{glyphtounicode}

\pdfglyphtounicode{f_f}{FB00}
\pdfglyphtounicode{f_f_i}{FB03}
\pdfglyphtounicode{f_f_l}{FB04}
\pdfglyphtounicode{f_i}{FB01}
\pdfgentounicode=1
%opening
\title{\LaTeXe Installation Instructions}
\author{Krishna Kumar\thanks{kks32@cam.ac.uk}}
\date{}
\begin{document}

\maketitle
\centering
\textbf{DVDs are available in the Grad Suite Computer room}
\flushleft
\section*{Windows OS: DVD-KCGS/Win/Latex/Jan2012/01}
\textbf{Basic MikTex (Tex package distribution) - compact / basic installer}\\
\begin{enumerate}
\item	Download Basic-MikTex 2.9 (32bit or 64bit) from \href{http://miktex.org/download}{http://miktex.org/download}
\item	Run the installer \href{http://docs.miktex.org/2.9/manual/ch02s02.html}{http://docs.miktex.org/2.9/manual/ch02s02.html}
\item	To add a new package go to Start >> All Programs >> MikTex 2.9 >> Maintenance (Admin) and choose Package Manager
\item	Select or search for packages to install
\end{enumerate}
or\\
\textbf{TeXLive package - full version}\\
\begin{enumerate}
\item	Download the TeXLive ISO (2.2GB) from \href{http://anorien.csc.warwick.ac.uk/mirrors/CTAN/systems/texlive/Images/texlive2013.iso}{http://anorien.csc.warwick.ac.uk/mirrors/CTAN/systems/texlive/Images/texlive2013.iso}
\item	Download WinCDEmu (if you don't have a virtual drive) from \href{http://wincdemu.sysprogs.org/download/WinCDEmu-3.6.exe}{http://wincdemu.sysprogs.org/download/WinCDEmu-3.6.exe}
\item	To install Windows CD Emulator follow the instructions at \href{http://wincdemu.sysprogs.org/tutorials/install/}{http://wincdemu.sysprogs.org/tutorials/install/}
\item	Right click the iso and mount it using the WinCDEmu as shown in \href{http://wincdemu.sysprogs.org/tutorials/mount/}{http://wincdemu.sysprogs.org/tutorials/mount/}
\item	Open your virtual drive and run setup.pl
\end{enumerate}

\textbf{TexStudio (Tex Editor)}\\
\begin{enumerate}
\item	Download TexStudio 2.6 from \href{http://texstudio.sourceforge.net/\#downloads}{http://texstudio.sourceforge.net/\#downloads} 
\item	Run the installer
\end{enumerate}

\section*{Mac OS X: DVD-KCGS/Mac/Latex/Jan2012/01}
\textbf{MacTex (Installs TexLive 2013)}\\
\begin{enumerate}
\item	Download the file from \href{http://mirror.ctan.org/systems/mac/mactex/MacTeX.pkg}{http://mirror.ctan.org/systems/mac/mactex/MacTeX.pkg}
\item	Extract and double click to run the installer. It does the entire configuration, sit back and relax.
\end{enumerate}

\textbf{TexStudio (Tex Editor)}
\begin{enumerate}
\item	Download TexStudio 2.6 from \href{http://texstudio.sourceforge.net/\#downloads}{http://texstudio.sourceforge.net/\#downloads} 
\item	Extract and Start
\end{enumerate}


\section*{Unix/Linux: DVD-KCGS/Unix/Latex/Jan2012/01}
\textbf{Tex Live 2013}\\
\textbf{Getting the distribution:}\\
\begin{enumerate}
\item	TexLive can be downloaded from \href{http://www.tug.org/texlive/acquire-netinstall.html}{http://www.tug.org/texlive/acquire-netinstall.html}. You might require wget to download through proxies.
\item	TexLive DVD is also available in the Grad Suite
\item	TexLive is provided by most operating system you can use (rpm,apt-get or yum) to get TexLive distributions (note: usually it's the old TexLive 2007) 
\end{enumerate}

\textbf{Installation}\\
\begin{enumerate}
\item	Mount the ISO file in the mnt directory\\

\begin{verbatim}mount -t iso9660 -o ro,loop,noauto /your/texlive2011.iso /mnt\end{verbatim}

\item	Get wget on your OS (use rpm, apt-get or yum install)
\item	Run the installer script install-tl.
\begin{verbatim}
	cd /your/download/directory
	./install-tl
\end{verbatim}
\item	Enter command `i' for installation

\item	Post-Installation configuration: \href{http://www.tug.org/texlive/doc/texlive-en/texlive-en.html\#x1-320003.4.1}{http://www.tug.org/texlive/doc/texlive-en/texlive-en.html\#x1-320003.4.1} 
\item	Set the path for the directory of TexLive binaries in your .bashrc file\\
\textbf{For 32Bit OS}\\
For Bourne-compatible shells such as bash, and using Intel x86 GNU/Linux and a default directory setup as an example, the file to edit might be \begin{verbatim}
edit $~/.bashrc file and add following lines
PATH=/usr/local/texlive/2011/bin/i386-linux:$PATH; export PATH 
MANPATH=/usr/local/texlive/2011/texmf/doc/man:$MANPATH; export MANPATH 
INFOPATH=/usr/local/texlive/2011/texmf/doc/info:$INFOPATH; export INFOPATH
\end{verbatim}
\textbf{For 64Bit}\\
\begin{verbatim}
edit $~/.bashrc file and add following lines
PATH=/usr/local/texlive/2011/bin/x86_64-linux:$PATH; export PATH 
MANPATH=/usr/local/texlive/2011/texmf/doc/man:$MANPATH; export MANPATH 
INFOPATH=/usr/local/texlive/2011/texmf/doc/info:$INFOPATH; export INFOPATH

\end{verbatim}

\end{enumerate}

\end{document}
